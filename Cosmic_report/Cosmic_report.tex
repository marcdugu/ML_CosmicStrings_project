\documentclass{article} 

%\documentclass{article}

% Language setting
\usepackage[english]{babel}

% Set page size and margins
% Replace `letterpaper' with `a4paper' for UK/EU standard size
\usepackage[letterpaper,top=2cm,bottom=2cm,left=3cm,right=3cm,marginparwidth=1.75cm]{geometry}

% Useful packages
\usepackage{amsmath}
\usepackage{graphicx}
\usepackage[x11names]{xcolor}
\usepackage[colorlinks=true, citecolor=magenta, linkcolor=black]{hyperref}
\usepackage{authblk}
\usepackage{blindtext}
\usepackage{multicol}




\title{\bf{Distinguishing between comsic strings and glitches using a neural network}}
\author[1]{Marlinde Drent}
\author[1]{Marc Duran Gutierrez}
\author[1]{Bo Ribbens}

\affil[1]{\it{Department of Physics, Utrecht University}}

\begin{document}
\maketitle

% Maybe add abstract?
\begin{abstract}
Your abstract.
\end{abstract}

\begin{multicols}{2}
\section{Introduction}
% A very general introduction
The first detected gravitational wave (GW) signal was GW150914\cite{PhysRevLett.116.061102} in 2015. 
Since then many technological advancements have been made, yet one thing that still plagues detectors is the presence glitches.    

\subsection{Cosmic strings}
% A deep dive into cosmic srings

\subsection{Machine learning}
% explain what machine learning is


\section{Method}
\subsection{Data}
% explain the data structure

\subsection{Neural network structure}
% explain the neural network structure (inlcude figure)

\section{Results}
\subsection{Accuracy}
% Show the accuracy of the neural network (include plot)

\subsection{Efficiency}
% Plot of loss and validation
% Some facts on the loss, training loops, time taken, etc.


\section{Discussion}
% Suggestions for further research -> how to improve the NN?


\section{Conclusion}
%TO DO:
% Conclude if NN works

\bibliographystyle{plain}
\bibliography{Cosmic_report}
\end{multicols}
\end{document}



%\begin{figure}
%\centering
%\includegraphics[width=0.25\linewidth]{}
%\caption{\label{fig:frog}This frog was uploaded via the file-tree menu.}
%\end{figure}